% !TEX program = xelatex

\documentclass{resume}
\usepackage[colorlinks,linkcolor=blue,anchorcolor=blue,citecolor=green,urlcolor=blue]{hyperref}

\begin{document}
\pagenumbering{gobble} % suppress displaying page number

\name{Qitong Wang}


\contactInfo{\href{mailto:qitongwang@hotmail.com}{\color{black} qitongwang@hotmail.com}}{(+86) 185-0212-7069}{}
\linespread{0.95}

% 
% Education 
% 

\section{Education}

\datedsubsection{\textbf{Fudan University}, School of Computer Science}{Sep 2016 -- Jun 2019}
\textit{Master} in Data Science\ \ \textit{Recommended without examinations}\ \ \textit{GPA} $3.53 / 4.0$\ \ 

\datedsubsection{\textbf{Fudan University}, School of Computer Science}{Sep 2012 -- Jun 2016}
% \textit{Bachelor} in Computer Science and Technology\ \ \textit{(Ranked in Top-}$0.05$\% \textit{in entrance exams)}\ \ \textit{GPA} $3.30 / 4.0$\ \ 
\textit{Bachelor} in Computer Science and Technology\ \ \textit{GPA} $3.30 / 4.0$\ \ 

\datedsubsection{\textbf{Stockholm University}, Department of Computer and Systems Sciences}{Sep 2014 -- Feb 2015}
\textit{Exchange Student with Full Scholarship}\ \ \textit{GPA} $3.40 / 4.0$\ \

% 
% Research Projects 
% 

\section{Research Experience}

\datedsubsection{\textbf{Interactive Chinese Phrase Mining}\\ Database and Massive Information Processing Laboratory at Fudan Univ.}{Sep 2017 -- Jun 2018}
\begin{itemize}
  \item Formalized problem, surveyed and experimented on related techniques, i.e., Chinese Word Segmentation and Phrase Mining
  \item Proposed an interactive Chinese phrase mining algorithm
  \item Implemented using Python/PyTorch, \textit{SkipGram} for embedding, 
  \textit{LSTM} for labeling, and \textit{active learning} for user interaction
\end{itemize}

\datedsubsection{\textbf{Mining of Moving Companion Patterns}\\ Database and Massive Information Processing Laboratory at Fudan Univ.}{Sep 2016 -- Aug 2017}
\begin{itemize}
  \item Characterized real-life data, formalized problem and surveyed related techniques
  \item Designed and implemented a distributed digest-based pruning algorithm, which pruned 99.5\% candidates 
  \item Optimized and refactored the proposed algorithm in Java using a two-level pruning structure with hash and digests, resulting 90\% promotion of running time
  \item Implemented using \emph{Java}, \emph{Scala}, \emph{Apache Spark}, \emph{Locality Sensitive Hashing (LSH)}, \emph{frequent pattern mining techniques}
\end{itemize}

\datedsubsection{\textbf{Development of Columnar File Format for Time-Series Data (TsFile)}\\ National Engineering Laboratory for Big Data Software at Tsinghua Univ.}{May 2016 -- Aug 2016}
\begin{itemize}
  \item Designed and developed the writing module of TsFile, a time series management system
  \item Studied literature of lossy compression encodings of data series, implemented and integrated into TsFile 
  Run-Length Encoding (RLE), Delta Compression, Piecewise Linear Approximation (PLA), Swinging Door Algorithm (SDA), and Fast Fourier transform (FFT)
  \item Analyzed codebase of Apache Impala, based on which implemented the distributed writing module of TsFile
  \item Implemented using \emph{C++}, \emph{Java}, \emph{Apache Thrift}, \emph{Apache Impala}
\end{itemize}

\datedsubsection{\textbf{Mining of Spatial Data}\\ Database and Massive Information Processing Laboratory at Fudan Univ.}{Nov 2015 -- Apr 2016}
\begin{itemize}
  \item Designed and implemented a distributed map matching algorithm using grid index and hidden markov model
  \item Implemented using \emph{Java}, \emph{Apache Hadoop}, \emph{Hidden Markov Model (HMM)}
\end{itemize}

\section{Internships}

\datedsubsection{\textbf{Research Assistant, The University of Hong Kong}\\ Advisor Prof. Reynold C.K. Cheng}{Dec 2018 -- Feb 2019}
\begin{itemize}
  \item Facilicate multimedia database queries with embedding techniques
  \item Implemented using \emph{C++}, \emph{Java}, \emph{Python}, \emph{Osmosis library of OpenStreetMap}
\end{itemize}

\datedsubsection{\textbf{Research Assistant, Tsinghua University}\\ Advisor Prof. Jianmin Wang and Chen Wang}{May 2016 -- Aug 2016}
\begin{itemize}
  \item Development of Columnar File Format for Time-Series Data (TsFile), deployed in Apache IoTDB
  \item Implemented using \emph{C++}, \emph{Java}, \emph{Apache Thrift}, \emph{Apache Impala}
\end{itemize}

% 
% Professional Experience
% 

\section{Professional Experience}
\datedsubsection{\textbf{Teaching Assistant, Fudan University}}{}
\datedsubsection{\textbf{} Probability and Statistics}{Sep 2017 -- Jan 2018}
\datedsubsection{\textbf{} Discrete Mathematics}{Sep 2017 -- Jan 2018}
\datedsubsection{\textbf{} Introduction to Databases}{Mar 2018 -- Jul 2018}
\datedsubsection{\textbf{} New Query Approach in Big Data Era}{Jul 2018}
\begin{itemize}
  \item Corrected homeworks, delivered recitations, arranged/designed/graded examinations
\end{itemize}

% 
% Publications
% 

\section{Publications}
\begin{itemize}  
  \item \textbf{Qitong Wang}, Peng Wang, Yuliang Zhao and Wei Wang. 
  An Efficient Approach to Mining Moving Companion Patterns. 
  The 35th Chinese National Database Conference (NDBC), Dalian, China, 2018
  \item Yuliang Zhao, \textbf{Qitong Wang}, Peng Wang, Wei Wang, and Zhiguo Yan. 
  A Distributed Approach of Accompany Vehicle Discovery. 
  The 6th International Conference on Applications and Techniques in Cyber Security and Intelligence (ATCI), Shanghai, China, 2018
  \item Zhongsheng Li, Qiuhong Li, Wei Wang, \textbf{Qitong Wang}, Fengbin Qi, Yimin Liu, and Peng Wang. 
  HDUMP: A Data Recovery Tool for Hadoop. 
  The 23th International Conference on Database Systems for Advanced Applications (DASFAA), Gold Coast, Australia, 2018
\end{itemize}

% 
% Awards
% 

\section{Awards}
\begin{itemize}
  \item 2019\ \ Outstanding Graduate of Shanghai City\ \ (\textit{Top} $5\%$) %\textit{Top} $5 / 108$
  \item 2017\ \ Tung OOCL Award at Fudan University\ \ (\textit{Top} $2\%$) %\textit{Top} $2 / 97$
  \item 2016/2013\ \  EMC Award at Fudan University\ \ (\textit{Top} $5\%$) %\textit{Top} $4 / 108$
  \item 2016\ \ Outstanding Graduate of Fudan University\ \ (\textit{Top} $10\%$) %\textit{Top} $5 / 108$
  \item 2019/2018/2016/2014/2013\ \ Outstanding Student Scholarship of Fudan University\ \ (\textit{Top} $30\%$)
\end{itemize}

% 
% Skills
% 

\section{Skills}
\begin{itemize}[parsep=0.5ex]
  \item Programming Skills: Java, Scala, Python, C/C++, SQL (PostgreSQL and MySQL)
  \item Open Source Library \& Frameworks: PyTorch, Weka, Apache Hadoop, Apache Spark, Apache Impala, Apache Thrift
  \item Language Skills: English (CET-4/6, TOEFL-102, GRE-326), Chinese (Native)
\end{itemize}

\end{document}
