% !TEX program = xelatex

\documentclass{resume}
\usepackage[colorlinks,linkcolor=blue,anchorcolor=blue,citecolor=green,urlcolor=blue]{hyperref}

\begin{document}
\pagenumbering{gobble} % suppress displaying page number

\name{Qitong WANG}

\contactInfo{\href{mailto:qitong.wang@etu.parisdescartes.fr}{\color{black} qitong.wang@etu.parisdescartes.fr}}{(+33) 6 69 52 81 38}{\href{https://qitongs.github.io/}{\color{black} qitongs.github.io}}
% \personalInfo{18-06-1993}{Chinese}{Male}{Single}
% \addressInfo{16 rue Emile Baudot, 91120 Palaiseau France}
\allInfo{18-06-1993}{Chinese}{Male}{Single}{16 rue Emile Baudot 91120 Palaiseau}
\linespread{0.95}

% 
% Education 
% 

\section{Education}

\datedsubsection{\textbf{Université de Paris}}{Paris, France}
\textit{PhD.} in Data Science, with Prof. \href{http://helios.mi.parisdescartes.fr/~themisp/home.html}{\color{black} Themis PALPANAS}\ \ \hfill Sep 2019 -- Jun 2023 (Est.)\ \

\datedsubsection{\textbf{Fudan University}}{Shanghai, China}
\textit{Master} in Data Science, with Prof. \href{http://www.cs.fudan.edu.cn/en/?page_id=2439}{\color{black} Wei WANG} and \href{http://homepage.fudan.edu.cn/pwang/}{\color{black} Peng WANG}\ \ \hfill Sep 2016 -- Jun 2019\ \ 

\datedsubsection{\textbf{Stockholm University}}{Stockholm, Sweden}
\textit{Exchange Undergraduate} in Computer Science\ \ \hfill Sep 2014 -- Feb 2015\ \

\datedsubsection{\textbf{Fudan University}}{Shanghai, China}
\textit{Bachelor} in Computer Science, with Prof. \href{http://homepage.fudan.edu.cn/pwang/}{\color{black} Peng WANG}\ \ \hfill Sep 2012 -- Jun 2016\ \

\datedsubsection{\textbf{No.2 Middle School of Yantai}}{Yantai, China}
\textit{High-School Student}\ \ \hfill Sep 2009 -- Jun 2012\ \

% 
% Research Projects 
% 

\section{Research Experiences}

\experiencesubsection{\textbf{Thesis: Machine Learning in Data Series Management and Analytics}}{Sep 2019 -- Present}
{diNo@LIPADE, Université de Paris}{Paris, France}
\begin{itemize}
  \item Exploiting deep neural models to facilicate data series indexing and query answering
  \item Exploring applications in Neurosciences and Astrophysics domains of data-series techniques, especially enhanced by deep neural models
  \item Implementing using \emph{Python/PyTorch}, \emph{C/C++}, various \emph{CNN/RNN achitectures}
\end{itemize}

\experiencesubsection{\textbf{Interactive Chinese Phrase Mining}}{Sep 2017 -- Jun 2018}
{Database Laboratory, Fudan University}{Shanghai, China}
\begin{itemize}
  \item Formalized problem, surveyed and experimented on related techniques, i.e., Chinese Word Segmentation and Phrase Mining
  \item Proposed an interactive Chinese phrase mining algorithm
  \item Implemented using \textit{Python/PyTorch}, \textit{SkipGram} for embedding, 
  \textit{LSTM} for labeling, and \textit{active learning} for user interaction
\end{itemize}

\experiencesubsection{\textbf{Thesis: Mining of Moving Companion Patterns}}{Sep 2016 -- Aug 2017}
{Database Laboratory, Fudan University}{Shanghai, China}
\begin{itemize}
  \item Characterized real-world data, formalized problem and surveyed related techniques
  \item Designed and implemented a distributed digest-based pruning algorithm, which pruned 99.5\% candidates 
  \item Optimized and refactored the proposed algorithm in Java using a two-level pruning structure with hash and digests, resulting 90\% promotion of running time
  \item Implemented using \emph{Java}, \emph{Scala}, \emph{Apache Spark}, \emph{Locality Sensitive Hashing (LSH)}, \emph{frequent pattern mining techniques}
\end{itemize}

\experiencesubsection{\textbf{Development of \href{https://github.com/thulab/tsfile}{\color{black} TsFile}, a Columnar File Format for Data Series}}
{May 2016 -- Aug 2016}
{National Big Data Software Laboratory, Tsinghua University}{Beijing, China}
\begin{itemize}
  \item Designed and developed the writing module of \href{https://github.com/thulab/tsfile}{\color{black} TsFile}, a time series management system
  \item Studied literature of lossy compression encodings of data series, implemented and integrated into \href{https://github.com/thulab/tsfile}{\color{black} TsFile} 
  Run-Length Encoding (RLE), Delta Compression, Piecewise Linear Approximation (PLA), Swinging Door Algorithm (SDA), and Fast Fourier transform (FFT)
  \item Analyzed codebase of Apache Impala, based on which implemented the distributed writing module of \href{https://github.com/thulab/tsfile}{\color{black} TsFile}
  \item Implemented using \emph{C++}, \emph{Java}, \emph{Apache Thrift}, \emph{Apache Impala}
\end{itemize}

\experiencesubsection{\textbf{Thesis: Mining of Spatial Data}}{Nov 2015 -- Apr 2016}
{Database Laboratory, Fudan University}{Shanghai, China}
\begin{itemize}
  \item Designed and implemented a distributed map-matching algorithm using grid index and hidden markov model
  \item Implemented using \emph{Java}, \emph{Apache Hadoop}, \emph{Hidden Markov Model (HMM)}
\end{itemize}

% 
% Internships 
% 

\section{Internships}

\experiencesubsection{\textbf{Research Assistant}}{Dec 2018 -- Feb 2019}
{With Prof. \href{https://i.cs.hku.hk/~ckcheng/}{\color{black} Reynold C.K. CHENG}, The University of Hong Kong}{Hong Kong, China}
\begin{itemize}
  \item Facilicated multimedia database queries with embedding techniques
  \item Implemented using \emph{C++}, \emph{Java}, \emph{Python}, \emph{Osmosis library of OpenStreetMap}
\end{itemize}

\experiencesubsection{\textbf{Research Assistant}}{May 2016 -- Aug 2016}
{With Prof. \href{http://www.thss.tsinghua.edu.cn/publish/soften/3131/2010/20101219100058471372347/20101219100058471372347_.html}{\color{black} Jianmin WANG} and \href{mailto:wang_chen@tsinghua.edu.cn}{\color{black} Chen WANG}, Tsinghua University}{Beijing, China}
\begin{itemize}
  \item Developed \href{https://github.com/thulab/tsfile}{\color{black} TsFile}, a Columnar File Format for Data Series, deployed in \href{http://iotdb.apache.org/}{\color{black} Apache IoTDB}
  \item Implemented using \emph{C++}, \emph{Java}, \emph{Apache Thrift}, \emph{Apache Impala}
\end{itemize}

% 
% Professional Experience
% 

\section{Professional Experiences}
\experiencesubsection{\textbf{Teaching Assistant}}{}
{School of Computer Science, Fudan University}{Shanghai, China}
\hspace{\parindent} Probability and Statistics\ \ \hfill Sep 2017 -- Jan 2018\ \

\hspace{\parindent} Discrete Mathematics\ \ \hfill Sep 2017 -- Jan 2018\ \

\hspace{\parindent} Introduction to Databases\ \ \hfill Mar 2018 -- Jul 2018\ \

\hspace{\parindent} New Query Approach in Big Data Era\ \ \hfill Jul 2018\ \

\begin{itemize}
  \item Corrected homeworks, delivered recitations, arranged/designed/graded examinations
\end{itemize}


% 
% Social Services 
% 

\section{Social Services}

\experiencesubsection{\textbf{Volunteer Caregiver}}{Feb 2015 -- Dec 2018}
{Sunshine Home}{Shanghai, China}
\begin{itemize}
  \item Entertained intellectual-disabled persons monthly
\end{itemize}

\experiencesubsection{\textbf{Volunteer Teacher}}{Jun 2013 -- Aug 2013}
{AIESEC Kilimanjaro}{Moshi, Tanzania}
\begin{itemize}
  \item Taught English and Mathematics at Mrupanga Primary School
  \item Raised funding for Tuleeni Orphans Home
\end{itemize}

% 
% Publications
% 

\section{Publications}
\begin{itemize}[parsep=0.5ex]
  \item \textbf{Qitong Wang}, Peng Wang, Yuliang Zhao and Wei Wang. 
  \textit{An Efficient Approach to Mining Moving Companion Patterns}. 
  The 35th Chinese National Database Conference (NDBC), Dalian, China, 2018
  \item Yuliang Zhao, \textbf{Qitong Wang}, Peng Wang, Wei Wang, and Zhiguo Yan. 
  \textit{A Distributed Approach of Accompany Vehicle Discovery}. 
  The 6th International Conference on Applications and Techniques in Cyber Security and Intelligence (ATCI), Shanghai, China, 2018
  \item Zhongsheng Li, Qiuhong Li, Wei Wang, \textbf{Qitong Wang}, Fengbin Qi, Yimin Liu, and Peng Wang. 
  \textit{HDUMP: A Data Recovery Tool for Hadoop}. 
  The 23th International Conference on Database Systems for Advanced Applications (DASFAA), Gold Coast, Australia, 2018
\end{itemize}

% 
% Awards
% 

\section{Awards}
\begin{itemize}[parsep=0.5ex]
  \item Outstanding Graduate of Shanghai City\ \ (\textit{Top} $5\%$) \ \ \hfill 2019
  \item Tung OOCL Award at Fudan University\ \ (\textit{Top} $2\%$) \ \ \hfill 2017
  \item EMC Award at Fudan University\ \ (\textit{Top} $5\%$) \ \ \hfill 2016/13
  \item Outstanding Graduate of Fudan University\ \ (\textit{Top} $10\%$) \ \ \hfill 2016
  \item Outstanding Student Scholarship of Fudan University\ \ (\textit{Top} $30\%$) \ \ \hfill 2019/18/16/14/13
\end{itemize}

% 
% Skills
% 

\section{Skills}
\begin{itemize}[parsep=0.5ex]
  \item Programming Skills: Python, C/C++, Java, Scala, SQL
  \item Open-Source Frameworks: PyTorch, Apache Hadoop, Apache Spark, and etc.
  \item Language Skills: English (CET-4/6, TOEFL-102, GRE-326), Chinese (Native)
\end{itemize}

\end{document}
